\documentclass[11pt,letterpaper]{article}
\usepackage{fullpage}
\usepackage[pdftex]{graphicx}
\usepackage{amsfonts,eucal,amsbsy,amsopn,amsmath}
\usepackage{url}
\usepackage[sort&compress]{natbib}
\usepackage{latexsym}
\usepackage{sectsty}
\usepackage[dvipsnames,usenames]{color}
\usepackage{multicol}
\usepackage{multirow}
\usepackage{caption}
\usepackage{fancyhdr}
\usepackage{times}
\renewcommand{\captionfont}{\small}
\allsectionsfont{\normalsize}
\pagestyle{fancy}
\lhead{}
\chead{}
\rhead{}
\lfoot{}
\cfoot{\thepage~of \pageref{lastpage}}
\rfoot{}
\renewcommand{\headrulewidth}{0pt}
\renewcommand{\footrulewidth}{0pt}
\usepackage{enumitem}
\setitemize{noitemsep,topsep=6pt,parsep=0pt,partopsep=0pt,leftmargin=1em}
\setenumerate{noitemsep,topsep=6pt,parsep=0pt,partopsep=0pt,leftmargin=1em}

\makeatletter
\renewcommand{\paragraph}{%
  \@startsection{paragraph}{4}%
  {\z@}{1.25ex \@plus 1ex \@minus .2ex}{-1em}%
  {\normalfont\normalsize\bfseries}%
}
\makeatother

\usepackage[compact]{titlesec}


\title{Team Name:  Final Report \\ CSE 481 N}
\author{List Authors Here}
\date{June 2019}

\begin{document}
\maketitle

\begin{abstract}
Write this last.  Include the key message of the document, distilled into one paragraph.  This is not a summary of what you did; it’s a summary that helps a reader decide whether to keep reading.  For this report, it makes sense to include a statement that this is a capstone class project, to give some context.
\end{abstract}

\section{Introduction}

Here is the problem we want to solve, why that problem is important (and to whom), and (at a high level) here is how we solve it.  This is also where you mention the most important related work, to help the reader locate the paper relative to the existing literature.  The most important related ideas that you build on should be cited here.  In some cases, a roadmap is a good way to close this section.

\subsection{High-Level Advice}
Assume your reader is intelligent and informed, but not necessarily an expert on the dataset, task, or problem you're working on.  Try to empathize with your reader and think about how to help them understand.  Your goal is not to impress the reader, but to make your ideas and argument clear to them.

\subsection{Answers to Some Frequently Asked Questions}
\begin{itemize}
\item Yes, you must use \LaTeX.
\item Aim to be precise and concise.  We expect most reports will be about four pages (not including references).  Do not expect us to read anything past page 5.
\item Do not tell the chronological story of your exploration this quarter; rather, think about what is most interesting to the reader.  Your decisions about how you order different parts, and how much room you give to each one, signal to the reader what's more important.
\item Negative and inconclusive results are reasonable to include, but be cautious about the conclusions you draw from them.  If there are still more experiments required to discern the reason for an observed difference between methods, say so, and consider suggesting what experiments might help clarify the situation, even if you don't have time to do them.
\end{itemize}



\section{Optional Section on Terminology and/or Notation}

Retitle or delete this section as appropriate.

\section{Technical Idea Section(s)}

Exposition of the new idea(s), with detailed discussion of how each relates to past work.  This is often the longest part of the paper, so you might want to break it into several sections, and include plenty of signposting so the reader doesn’t get lost.  This is where you will discuss as much related work as you can (without breaking the flow).  You can also hint at future directions or explicitly leave open questions to future work, but use sparingly or the reader will get the impression that you’re writing about work in progress.  Think carefully about how to structure this section to make it easy to understand.

Consider using figures and tables to present information as clearly as possible.

\begin{figure}
\begin{center}You'll likely use a command like \texttt{includegraphics} to include images here.\end{center}
\caption{\label{fig:foo} Consider using figures and tables to present information as clearly as possible.}
\end{figure}

Any time you include a figure or table, like Figure~\ref{fig:foo}, make sure it's referenced in the text.

\subsection{Subteam Effort [list name(s) here]}

By default, your team will share the grade for this report.  However, you may optionally label technical ideas sections or subsections with names of the people responsible for them (i.e., when it's a strict subset of the team).



\section{Empirical Section(s)} 

These sections support the claims of the paper by presenting experiments or other data analysis.  When these open new questions for future work, you can say that.  Try to tease apart (1) the experimental design (including justification for why you're doing the comparisons you're doing---what is the question you want to answer with this experiment?), (2) details of the datasets and evaluation scores used in your paper, (3) the implementation details for the systems under consideration, (4) the results, and (5) the interpretation of those results.  Often these get muddled up.  In different situations, different orderings of (1-3) make sense.  Separating (4) and (5) is especially important.  A common pitfall is to present a large table of numbers without pointing out the most important things to observe in the table.  Another common pitfall is to jump to broad conclusions based on observed differences in performance; be cautious about the conclusions that you draw.

\begin{table}
\begin{center}You'll probably use a \texttt{tabular} environment here.\end{center}
\caption{\label{tab:foo} Consider using figures and tables to present information as clearly as possible.}
\end{table}

Any time you include a figure or table, like Table~\ref{tab:foo}, make sure it's referenced in the text.

\subsection{Subteam Effort [list name(s) here]}

By default, your team will share the grade for this report.  However, you may optionally label empirical sections or subsections with names of the people responsible for them (i.e., when it's a strict subset of the team).


\section{Related Work}

Related work that didn't fit the flow of the paper earlier, but which you strategically want to mention, can be discussed in a short ``related work'' section near the end.  I strongly advise against a related work section immediately after the introduction; that’s like putting up a wall between your reader and the important new idea, but in some settings it may be necessary to to situate the reader in some very closely related work so your ideas make sense.

Make sure you understand the difference between citing work as a noun phrase (\texttt{citet}) and using a parenthetical (\texttt{citep}).  The first one looks like \citet{clark:2018:iui}, and the second one is different \citep{lin:2018:ghtc}.


\section{Optional Section(s) on Limitations and/or Future Work}

Retitle or delete this section as appropriate.

\section{Conclusion}

A clear statement of what we know now that we have read this document, but didn’t know before.  This is not the place for future work or grand philosophical statements.


\section*{Acknowledgments}

Acknowledge anyone who gave you helpful comments or suggestions, including your classmates and people outside the class who met with you about the project.

Put your bibtex references in a file called \texttt{refs.bib}.  The citation to \citet{press:2018:attention} exemplifies the style I prefer for arXiv papers; when a paper has been published in a peer reviewed forum, cite that version instead of the arXiv version.

\bibliographystyle{plainnat}
\bibliography{refs}



\label{lastpage}
\end{document}
